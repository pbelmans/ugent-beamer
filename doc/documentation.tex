\documentclass[a4paper,10pt]{article}
\usepackage{amsthm}
\usepackage{booktabs}
\usepackage{graphicx}
\usepackage{subfig}
\usepackage{tabulary}
\usepackage{tikz}
\usepackage{todonotes}
\usepackage{xcolor}

\usepackage[backend=biber]{biblatex}
\bibliography{bibliography.bib}

\usepackage[T1]{fontenc}
\usepackage[charter]{mathdesign}
\usepackage[scaled]{beramono,berasans}
\usepackage{microtype}

\usepackage{listings}
\lstset{
  basicstyle = \ttfamily,
  keywordstyle = \color{ugentblue},
  identifierstyle =,
  commentstyle = \color{olive},
  stringstyle = \color{ugentyellow},
  showstringspaces = false,
  frame = single,
  language = [LaTeX]TeX
}

\usepackage[bookmarks, colorlinks, linkcolor = ugentblue, citecolor=ugentblue, urlcolor=ugentblue, pdftitle={A beamer presentation class theme for the University of Ghent}, pdfauthor={Pieter Belmans}]{hyperref}

\newtheorem{note}{Note}
\newlength{\figurewidth}
\newlength{\figureheight}

\newcommand{\changefont}[3]{\fontfamily{#1}\fontseries{#2}\fontshape{#3}\selectfont}


\title{A \texttt{beamer} presentation class theme for the University of Ghent}
\author{Pieter Belmans\thanks{e-mail: \href{mailto:pieter.belmans@ugent.be}{\nolinkurl{pieter.belmans@ugent.be}}. Comments and suggestions always welcome.}}
\date{Version 1 \\ \today}

% Pantone 534: global
\definecolor{ugentblue}{RGB}{10,30,96}
% Pantone 130: generic colour
\definecolor{ugentyellow}{RGB}{255,179,0}


% Pantone 103c: Faculty of Literature & Philosophy
\definecolor{ugent-lw}{RGB}{198,173,15}
% Pantone 485c: Faculty of Law
\definecolor{ugent-re}{RGB}{204,12,0}
\definecolor{ugent-re2}{RGB}{216,30,0} % not the actual 485, but looks closer
% Pantone 292c: Faculty of Science
\definecolor{ugent-we}{RGB}{117,178,221}
% Pantone 702: Faculty of Medicine and Health Sciences
\definecolor{ugent-ge}{RGB}{214,96,109}
% Pantone 272c: Faculty of Engineering and Architecture
\definecolor{ugent-tw}{RGB}{94,104,196} % but there are other options
% Pantone 556c: Faculty of Economics and Business Administration
\definecolor{ugent-eb}{RGB}{122,168,145}
% Pantone 2583c: Faculty of Veterinary Medicine
\definecolor{ugent-di}{RGB}{148,79,165}
% Pantone 1655c: Faculty of Psychology and Educational Sciences
\definecolor{ugent-pp}{RGB}{249,86,2}
% Pantone 3262c: Faculty of Bioscience Engineering
\definecolor{ugent-la}{RGB}{0,193,181}
% Pantone purple: Faculty of Pharmaceutical Sciences
\definecolor{ugent-fw}{RGB}{165,68,153}
% Pantone 375c: Faculty of Political and Social Sciences
\definecolor{ugent-ps}{RGB}{140,214,0}


\newcommand\colorsquare[2]{\colorbox{#2}{\rule{0mm}{#1}\rule{#1}{0mm}}}
\newcommand\colorsquareseries[2]{%
  \colorsquare{#1}{#2}%
  \colorsquare{#1}{#2!75!white}%
  \colorsquare{#1}{#2!50!white}%
  \colorsquare{#1}{#2!15!white}%
  \colorsquare{#1}{#2!10!white}%
  \colorsquare{#1}{#2!5!white}%
}

\begin{document}

\maketitle

\begin{abstract}
  This is a theme for the \LaTeX\ \texttt{beamer} class for presentations. It has been designed to match the official UGent templates for Microsoft PowerPoint\textsuperscript{\textregistered} while doing away with some of the shortcomings of it within the \texttt{beamer} framework. If you would like to give a scientific presentation with all the benefits you get from using \LaTeX, this theme is for you.\footnote{Note that there exists another \texttt{beamer} theme for the University of Ghent (or at least the Engineering faculty) designed by \textsc{Harald Devos} \cite{devos-theme}. In contrast to this theme, it is not designed with compliance to the official themes in mind and it is not tailored to easy integration with the different faculties.}
\end{abstract}


\section{Installation}

\subsection{Requirements}
  The theme is an addendum for the \LaTeX\ \texttt{beamer} class, hence it is assumed that you have a running \LaTeX{} installation together with the \texttt{beamer} package \cite{beamer-repository} and its dependencies. If not already the case, it will also certainly be a good idea to make yourself familiar with the \texttt{beamer} class. The excellent manual \cite{beamer-manual} is will serve you well.

\subsection{Getting the package}

The package is distributed via GitHub \cite{ugent-beamer}. You can get the latest releases there, as well as additional information about the package. More specifically the Downloads page at \url{https://github.com/pbelmans/ugent-beamer/downloads} will serve most of your needs. If you would like to stay on the bleeding edge of development, you can get access to the Git repository by
\begin{lstlisting}
$ git clone git@github.com:pbelmans/ugent-beamer.git
\end{lstlisting}
You are free to improve and extend the implementation, preferably by forking and merging to keep development centralized.


\subsection{First run}
Once you have the files, all that is required for the theme to work is putting the files into a directory where \LaTeX{} can find them. If you would like to install it for one user only, this would typically be \lstinline{~/texmf/} (on Unix). For all other options please refer to the documentation of your \LaTeX{} package manager.

You can check if \LaTeX{} finds the files in its directories by, for example, looking for the file \lstinline!beamerfontthemeUniversiteitGent.sty!,
\begin{lstlisting}
$ kpsewhich beamerfontthemeUniversiteitGent.sty
\end{lstlisting}

If everything seems okay, you can check whether you can actually produce a presentation with the UGent theme by either creating a minimal test file (see listing \ref{listing:minimal-example}) or by compiling the \LaTeX-document provided in the \lstinline!example/! folder of this package. 


\begin{figure}
  \centering
  \setlength{\figurewidth}{6cm}
  \subfloat[][The first two pages, no faculty specified.]{%
    \frame{\includegraphics[width=\figurewidth]{figures/slide-1}}%
    \qquad
    \frame{\includegraphics[width=\figurewidth]{figures/slide-2}}%
  }

  \subfloat[][The first two pages, with \lstinline|faculty=lw| and \lstinline|language=english| specified.]{%
    \frame{\includegraphics[width=\figurewidth]{figures/slide-faculty-1}}%
    \qquad
    \frame{\includegraphics[width=\figurewidth]{figures/slide-faculty-2}}%
  }
\end{figure}

\begin{note}
  You might need to install more \LaTeX{} packages when running the provided example file (e.g., the \texttt{lipsum} package).
\end{note}

\begin{lstlisting}[float,caption={A minimalistic test file for the UA Beamer theme.},captionpos=b,label=listing:minex,abovecaptionskip=\bigskipamount]
\documentclass{beamer}

\usetheme{UniversiteitGent}

\begin{document}

\begin{frame}
  \frametitle{Example}
  Hello world!
\end{frame}

\end{document}
\end{lstlisting}


\section{Theme options}

The theme comes with several options, all of which can be given in a comma-separated list like in
\begin{lstlisting}
\usetheme[options]{UniversiteitGent}
\end{lstlisting}

\begin{center}
\begin{tabulary}{12cm}{p{3cm}L}
  \toprule
  \lstinline|faculty=| & Every faculty can (or should) use its own header, as explained at \url{http://www.ugent.be/nl/werken/organisatie/huisstijl/kantoor/presentaties.htm} (only available to UGent members). This option needs a value, see Table~\ref{table:faculty-values} for an overview. Providing this value will only change the header, but it is related to the options \lstinline|language| and \lstinline|usecolours|. \\\midrule
  \lstinline|language=| & The official template provides both an English and a Dutch header for each faculty, the default is taken to be the \emph{Dutch version}. In case you want to use the English version, provide the option \lstinline|language=english|. Remark that the default corresponds to \lstinline|language=dutch|, but it is not necessary to provide this. This valued approach is written with future extension in mind. \\\midrule
  \lstinline|usecolors| & If you have provided a value for \lstinline|faculty=| and specify this option the faculty color (which is featured in the header) will be used in other parts of the presentation, e.g.\ in an \lstinline|alertblock|. Otherwise the standard color \lstinline|ugentyellow| is used (see Section~\ref{section:colors}). \\\midrule
  \lstinline!framenumber! &  The \lstinline!framenumber! option makes sure that the number of the current frame is displayed. \\\midrule
  \lstinline!totalframenumber! & With \lstinline!framenumber! turned on, the \lstinline!totalframenumber! option makes sure that the total number of frames is displayed alongside with the current frame number.\\\bottomrule
  \end{tabulary}
\end{center}

\begin{note}
  The \texttt{beamer} option \lstinline!compress! is respected in the sense that, if it is provided, header and footer will take less space such that there is more space for actual frame content\todo{implement this}. The navigational symbols are always hidden, regardless of this option.
\end{note}

\begin{note}
  Specifying a wrong value for the \lstinline|language=| option, or using \lstinline|usecolors| without specifying a value for \lstinline|faculty=| will result in errors.
\end{note}


\section{Colors}

\subsection{Background on the official colors}

\begin{note}
  If you don't care about the technical details, you can skip this and directly go to subsection \ref{subsection:usingthecolors}.
\end{note}

The two official colors of the UGent, a blue and a red tone, are originally given in PMS format, a \emph{proprietary} format issued by the Pantone Inc.\ corporation. One particular feature of the PMS color format is that it is, as opposed to RGB or CMYK, \emph{device independent}. This means that the definition of the color does not consist of instructions such as \enquote{put 37\% red, 12\% green, and 45\% blue and mix it all together}, but really is a concise description of what the final result, be it printed or displayed, physically looks like on the medium under determined ambient light conditions.

The PMS format is richer than RGB in the sense that it can embrace fluorescence effects, gold or silver shine, special coatings (matte and brilliant), in general everything that has to do with the actual appearance of the color on the medium. It comes thus as no surprise that there is no (exact) mapping between the PMS and RGB color spaces; more specifically: the mapping -- if it exists -- is device-dependent.

The \emph{only} way to get a perfect PMS 1955 red, for example, is to have a the plot file ready with the (proprietary) PMS color information in it, and have it printed on a PMS-ready printer which needs to be filled the special PMS 1955 ink beforehand. This process is eponymous for colors which are not composed of different types of (yellow, blue, red) inks: they are called \emph{single-spot colors}, and most Pantone\textsuperscript{\textregistered} colors are of such kind.


\subsection{Conversion to RGB/CMYK}

When the computer screen or any other non-PMS-ready medium is the primary output source, having such rigorous rules may be rather obstructive. To overcome such restrictions, many vendors of graphical software try to translate PMS into RGB/CMYK by making use of special monitor color calibration data (e.g., the ICC color profiles, \lstinline!.icc!). Unfortunately, if files containing PMS color information are displayed (printed) with programs (printers) which do not support PMS mechanisms (which is most often the case in non-professional environments), the color output will look disturbed. On top of that, and surprisingly for the novice, the output will look disturbed in different ways on different screens (printers) because of the inherent device-dependence of PMS.

%To overcome the undesired side-effects of working with the official UA corporate design in a non-PMS environment (like most beamers and computer screens should provide),
That is why this theme aims to use non-PMS versions of the UA's official colors, including modified versions of the logos.\todo{what should I do?}

Now, as explained above there is no device-independent conversion between the original PMS directives and the CMYK color space; several tables exist which are valid for different work environments. On the official sites of the UGent \cite{KAN::}, you'll find particular RGB and CMYK values, other resources, e.g., \cite{::TDC}, provide other numbers (see table \ref{table:reds-blues}).

For the sake of consistency between RGB and CMYK values, the UGent \texttt{beamer} theme uses the CMYK values for PMS 302 and PMS 1955 provided on \cite{::TDC} for reference (see table \ref{table:reds-blues}, last column, and table \ref{table:allcolors}).

\subsection{Using the colors}\label{subsection:usingthecolors}


Getting a consistent look-and-feel throughout your presentations requires sticking to a particular style scheme, most of which is being implemented in the \texttt{beamer} style file already. One particular aspect, though, can only be controlled by the user, and that is the colors that are used in the running text, tables, and figures.

Although essentially consistent of only two colors, within \texttt{beamer} referred to as \lstinline!ugentblue! and \lstinline!ugentyellow!, provide more diversity than one might expect and should be  \emph{exclusively} used  in all the slides. The user should be aware that this directive includes \emph{tables, figures, and graphics of most kinds}. For an example of usage see figure \ref{fig:coloredgraphs}.

\begin{figure}
  \setlength\figurewidth{10cm}
  \setlength\figureheight{5cm}
  \centering
  %\input{figures/graphs.tikz}
  \caption{Example usage of the official colors within a set of graphs, using also the exception color.}
  \label{figure:coloredgraphs}
\end{figure}

If you need to emphasize a particular aspect in your slides (graphs, tables), you can (within \texttt{beamer}) use the \lstinline!\alert{}! macro (e.g., \lstinline!\alert{This is alerted text.}!). For the situation where something needs to stick out in a pie chart, for example, where the ordinary colors (UA blue and red) have been used up already, a third color has been added (see table \ref{table:allcolors}). It is to be used scarcely and strictly for highlighting purposes (see, for example, figure \ref{fig:coloredgraphs}).


\begin{table}
\setlength{\figurewidth}{12mm}
\centering
\begin{tabulary}{\textwidth}{LCCC}\toprule
Color sample & \colorsquare{\figurewidth}{ugentblue!100!white} & \colorsquare{\figurewidth}{ugentblue!75!white} & \colorsquare{\figurewidth}{ugentblue!50!white}\\
CMYK & (100, 30, 0, 62) & (75, 23, 0, 47) & (50, 15, 0, 31)\\
RGB  & (0, 61, 100)     & (40, 128, 158)  & (96, 166, 191)\\
\texttt{beamer} name & \lstinline|ugentblue!100!white| & \lstinline|ugentblue!75!white| & \lstinline|ugentblue!50!white| \\[3mm]
Color sample & \colorsquare{\figurewidth}{ugentblue!25!white} & \colorsquare{\figurewidth}{ugentblue!10!white} & \colorsquare{\figurewidth}{ugentblue!5!white}\\
CMYK & (25, 8, 0, 16)  & (10, 3, 0, 6)   & (5, 2, 0, 3)\\
RGB  & (166, 209, 222) & (218, 235, 242) & (235, 245, 247)\\
\texttt{beamer} name & \lstinline!ugentblue25! & \lstinline!ugentblue10! & \lstinline!ugentblue5!\\\midrule

Color sample & \colorsquare{\figurewidth}{ugentyellow!100!white} & \colorsquare{\figurewidth}{ugentyellow!75!white} & \colorsquare{\figurewidth}{ugentyellow!50!white}\\
CMYK & (0, 100, 54, 46) & (0, 75, 41, 35) & (0, 50, 27, 23)\\
RGB  & (161, 0, 64)     & (184, 46, 101)  & (207, 103, 145)\\
\texttt{beamer} name & \lstinline|ugentyellow!100!white| & \lstinline|ugentyellow!75!white| & \lstinline|ugentyellow!50!white| \\[3mm]
Color sample & \colorsquare{\figurewidth}{ugentyellow!25!white} & \colorsquare{\figurewidth}{ugentyellow!10!white} & \colorsquare{\figurewidth}{ugentyellow!5!white}\\
CMYK & (0, 25, 14, 12)  & (0, 10, 5, 5)   & (0, 5, 3, 2)\\
RGB  & (232, 174, 197) & (245, 220, 230) & (250, 237, 242)\\
\texttt{beamer} name & \lstinline|ugentyellow!25!white| & \lstinline|ugentyellow!10!white| & \lstinline|ugentyellow!5!white| \\\midrule

%Color sample & \colorsquare{\figurewidth}{vividbrown} & \colorsquare{\figurewidth}{vividbrown75} & \colorsquare{\figurewidth}{vividbrown50}\\
%CMYK & (0, 28, 67, 16) & (0, 20, 48, 12) & (0, 13, 31, 8)\\
%RGB  & (215, 154, 70) & (225, 179, 116)  & (235, 205, 163)\\
%\texttt{beamer} name & \lstinline!vividbrown100! & \lstinline!vividbrown75! & \lstinline!vividbrown50!\\[3mm]
%Color sample & \colorsquare{\figurewidth}{vividbrown25} & \colorsquare{\figurewidth}{vividbrown10} & \colorsquare{\figurewidth}{vividbrown5}\\
%CMYK & (0, 6, 15, 4) & (0, 2, 6, 2) & (0, 1, 3, 1)\\
%RGB  & (245, 230, 209)  & (251, 245, 237) & (253, 250, 246)\\
%\texttt{beamer} name & \lstinline!vividbrown25! & \lstinline!vividbrown10! & \lstinline!vividbrown5!\\\bottomrule
\end{tabulary}

\caption{Overview over the two main colors used in theme as well as the third color for exceptions and highlighting.}
\label{table:allcolors}
\end{table}

%\begin{table}
%\begin{tabular*}{\textwidth}{cc}
%\toprule
%Category 1 &
%\colorbox{ugentblue100}{\rule{0mm}{5mm}\rule{5mm}{0mm}}
%\colorbox{ugentblue75}{\rule{0mm}{5mm}\rule{5mm}{0mm}}
%\colorbox{ugentblue50}{\rule{0mm}{5mm}\rule{5mm}{0mm}}
%\colorbox{ugentblue25}{\rule{0mm}{5mm}\rule{5mm}{0mm}}
%\colorbox{ugentblue10}{\rule{0mm}{5mm}\rule{5mm}{0mm}}
%\colorbox{ugentblue5}{\rule{0mm}{5mm}\rule{5mm}{0mm}}\\
%&
%\colorbox{ugentyellow100}{\rule{0mm}{5mm}\rule{5mm}{0mm}}
%\colorbox{ugentyellow75}{\rule{0mm}{5mm}\rule{5mm}{0mm}}
%\colorbox{ugentyellow50}{\rule{0mm}{5mm}\rule{5mm}{0mm}}
%\colorbox{ugentyellow25}{\rule{0mm}{5mm}\rule{5mm}{0mm}}
%\colorbox{ugentyellow10}{\rule{0mm}{5mm}\rule{5mm}{0mm}}
%\colorbox{ugentyellow5}{\rule{0mm}{5mm}\rule{5mm}{0mm}}\\\midrule
%Category 2 &
%\colorbox{col22}{\rule{0mm}{5mm}\rule{5mm}{0mm}}
%\colorbox{col22!75!white}{\rule{0mm}{5mm}\rule{5mm}{0mm}}
%\colorbox{col22!50!white}{\rule{0mm}{5mm}\rule{5mm}{0mm}}
%\colorbox{col22!25!white}{\rule{0mm}{5mm}\rule{5mm}{0mm}}
%\colorbox{col22!10!white}{\rule{0mm}{5mm}\rule{5mm}{0mm}}
%\colorbox{col22!5!white}{\rule{0mm}{5mm}\rule{5mm}{0mm}}\\
%&
%\colorbox{col24}{\rule{0mm}{5mm}\rule{5mm}{0mm}}
%\colorbox{col24!75!white}{\rule{0mm}{5mm}\rule{5mm}{0mm}}
%\colorbox{col24!50!white}{\rule{0mm}{5mm}\rule{5mm}{0mm}}
%\colorbox{col24!25!white}{\rule{0mm}{5mm}\rule{5mm}{0mm}}
%\colorbox{col24!10!white}{\rule{0mm}{5mm}\rule{5mm}{0mm}}
%\colorbox{col24!5!white}{\rule{0mm}{5mm}\rule{5mm}{0mm}}\\
%&
%\colorbox{col21}{\rule{0mm}{5mm}\rule{5mm}{0mm}}
%\colorbox{col21!75!white}{\rule{0mm}{5mm}\rule{5mm}{0mm}}
%\colorbox{col21!50!white}{\rule{0mm}{5mm}\rule{5mm}{0mm}}
%\colorbox{col21!25!white}{\rule{0mm}{5mm}\rule{5mm}{0mm}}
%\colorbox{col21!10!white}{\rule{0mm}{5mm}\rule{5mm}{0mm}}
%\colorbox{col21!5!white}{\rule{0mm}{5mm}\rule{5mm}{0mm}}\\
%&
%\colorbox{col42}{\rule{0mm}{5mm}\rule{5mm}{0mm}}
%\colorbox{col42!75!white}{\rule{0mm}{5mm}\rule{5mm}{0mm}}
%\colorbox{col42!50!white}{\rule{0mm}{5mm}\rule{5mm}{0mm}}
%\colorbox{col42!25!white}{\rule{0mm}{5mm}\rule{5mm}{0mm}}
%\colorbox{col42!10!white}{\rule{0mm}{5mm}\rule{5mm}{0mm}}
%\colorbox{col42!5!white}{\rule{0mm}{5mm}\rule{5mm}{0mm}}\\\midrule
%Alert color &
%\colorbox{col32}{\rule{0mm}{5mm}\rule{5mm}{0mm}}
%\colorbox{col32!75!white}{\rule{0mm}{5mm}\rule{5mm}{0mm}}
%\colorbox{col32!50!white}{\rule{0mm}{5mm}\rule{5mm}{0mm}}
%\colorbox{col32!25!white}{\rule{0mm}{5mm}\rule{5mm}{0mm}}
%\colorbox{col32!10!white}{\rule{0mm}{5mm}\rule{5mm}{0mm}}
%\colorbox{col32!5!white}{\rule{0mm}{5mm}\rule{5mm}{0mm}}\\\bottomrule
%\end{tabular*}
%\caption{Color table.}
%\end{table}

%\colorbox{col11}{\rule{0mm}{5mm}\rule{5mm}{0mm}}
%\colorbox{col12}{\rule{0mm}{5mm}\rule{5mm}{0mm}}
%\colorbox{col13}{\rule{0mm}{5mm}\rule{5mm}{0mm}}
%\colorbox{col14}{\rule{0mm}{5mm}\rule{5mm}{0mm}}\\
%\colorbox{col21}{\rule{0mm}{5mm}\rule{5mm}{0mm}}
%\colorbox{col22}{\rule{0mm}{5mm}\rule{5mm}{0mm}}
%\colorbox{col23}{\rule{0mm}{5mm}\rule{5mm}{0mm}}
%\colorbox{col24}{\rule{0mm}{5mm}\rule{5mm}{0mm}}\\
%\colorbox{col31}{\rule{0mm}{5mm}\rule{5mm}{0mm}}
%\colorbox{col32}{\rule{0mm}{5mm}\rule{5mm}{0mm}}
%\colorbox{col33}{\rule{0mm}{5mm}\rule{5mm}{0mm}}
%\colorbox{col34}{\rule{0mm}{5mm}\rule{5mm}{0mm}}\\
%\colorbox{col41}{\rule{0mm}{5mm}\rule{5mm}{0mm}}
%\colorbox{col42}{\rule{0mm}{5mm}\rule{5mm}{0mm}}
%\colorbox{col43}{\rule{0mm}{5mm}\rule{5mm}{0mm}}
%\colorbox{col44}{\rule{0mm}{5mm}\rule{5mm}{0mm}}

\printbibliography

\end{document}
